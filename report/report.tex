\documentclass[a4paper,11pt]{article}
\usepackage{latexsym,amssymb,enumerate,amsmath,epsfig,amsthm}
\usepackage[margin=1in]{geometry}
\usepackage{setspace,color}
\usepackage{graphics}
\usepackage{subfigure}

%\newcommand{\x}{\mathbf{x}}
%\newcommand{\y}{\mathbf{y}}
%\newcommand{\bv}{\mathbf{v}}
%\newcommand{\n}{\mathbf{n}}
%\newcommand{\colored}[1]{\textcolor{red}{#1}}
%\newtheorem{thm}{Theorem}[section]
%\newtheorem{prop}{Proposition}[section]
%\newtheorem{obser}{Observation}[section]
%\newtheorem{corollary}{Corollary}[section]

%\doublespacing

\title{Modeling and Controlling of an Inverted Pendulum System with Simulation}
\author{
Xingzhou CHEN \thanks{Department of Electronic and Computer Engineering, the Hong Kong University of Science and Technology, Clear Water Bay, Hong Kong. Email: {\bf xchenfk@connect.ust.hk}}  
}

\markboth{S. Leung}{MATH5311 Final Project}
\pagestyle{myheadings}

\begin{document}
\thispagestyle{plain}
\maketitle

\begin{abstract}
In this project, I model an inverted pendulum system, which can be formulated as a multi-variable second-order ordinary differential equation. 
In addition, I use the different numerical method to simulate how the system would change dynamically. Further, to keep the system in balance,
I apply "linear quadratic regulator"(LQR) method and get a numerical solution of the feedback gain. The result demonstrates the convergence of
our numerical scheme.

\end{abstract}

\section{Introduction}
Dynamic balance control problem is a fundamental problem in control theory. With the rise of drones and robotic dogs, how to maintain dynamic balance 
in complex dynamic systems has become a research frontier. In order to solve these problems, we need to formualte a model first, then approximate 
and simulate the system, and select the appropriate algorithm for balance control. \\
In this project, I apply "linear quadratic regulator"(LQR) to 
the inverted pendulum system, which is an absolute-unstable, high-order, multi-variables and strong-coupled system. In second section, the inverted pendulum 
system and the LQR algorithm are briefly introduced. With numerical methods, the third section shows the simulation of the dynamic system and the solution of 
feedback gain in the LQR algorithm. In forth section, I present the results and discuss them.
\begin{figure}[!htb]
	\centerline{
		\psfig{figure=picture/Inverted.jpeg,width=2.0in}
	}
	\caption{Inverted pendulum system}
	\label{Figure:Cart with inverted pendulum}
\end{figure}

\section{Inverted pendulum system and Linear quadratic regulator}
\subsection{Inverted pendulum system}
Consider a cart with an inverted pendulum hinged on top of it as shown in Figure.\ref{Figure:Cart with inverted pendulum}. 
For simplicity, the cart and the pendulum are assumed to move in only one plane, and the friction, the mass of the stick, and the gust of wind are disregarded.
The problem is to maintain the pendulum at the vertical position.\cite{chen1999linear} And the physical symbol is shown in Table \ref{tab:symbol}.
\begin{table}[!htbp]
	\centering
	\caption{\textbf{the symbol of physical model}} \label{tab:symbol}
	\begin{tabular}{lll}
	\multicolumn{1}{l|}{Symbol} & Physical meaning            & Value                     \\ \hline
	\multicolumn{1}{l|}{$M$}      & the quality of the cart              &  0.5      \\
	\multicolumn{1}{l|}{$m$}      & the quality of the pendulum           & 0.2       \\
	\multicolumn{1}{l|}{$l$}      & the length of the stick                 & 0.3     \\
	\multicolumn{1}{l|}{$I$}      & the moment of inertia of the pendulum and stick  0.006 \\
	\multicolumn{1}{l|}{$c$}      & the rotational friction coefficient of the stick  \\
	\multicolumn{1}{l|}{$F$}      & the force applied to the cart                   \\
	\multicolumn{1}{l|}{$x$}      & the displacement of the cart                      \\
	\multicolumn{1}{l|}{$\theta$}	   & the angle between the normal and the stick      \\
	\multicolumn{1}{l|}{$g$}	   & acceleration of gravity                         & 10      
	\end{tabular}
\end{table}
\\
Let $P$ and $N$ be, respectively, the horizontal and vertical forces exerted by the cart on the pendulum as shown. The application of Newton’s law to the linear movements yields
\begin{figure}[!hbt]
	\centering
	\mbox{
	(a)\subfigure{\psfig{figure=picture/force_1.png,width=2.6in}} \quad\quad
	(b)\subfigure{\psfig{figure=picture/force_2.png,width=1.2in}}
	}
	\vspace{15pt}
	\caption{The horizontal and vertical forces}
	\label{Figure:The horizontal and vertical forces}
\end{figure}
\\
In the horizontal direction, the cart and the pendulum satisfies the following equation
\begin{equation}
	M \ddot{x} = F-N.	\label{Eqn:1}
\end{equation}
\begin{equation}
	N = m{(x+l\sin{\theta})}'' = m\ddot{x} + ml\ddot{\theta}\cos{\theta} - ml\dot{\theta}^2\sin{\theta}. \label{Eqn:2} 
\end{equation}
In the vertical direction, the pendulum satisfies the following equation
\begin{equation}
	P-mg = -ml(1-\cos{\theta})'' = -ml\ddot{\theta}\sin{\theta}-ml\dot{\theta}^2\cos{\theta}.	\label{Eqn:3}
\end{equation}
The torque balance equation for the pendulum and the stick is as follows
\begin{equation}
	Pl\sin{\theta}-Nl\cos{\theta} - c\dot{\theta} = I\ddot{\theta}.	\label{Eqn:4}
\end{equation}
Combined the equation (\ref{Eqn:1})(\ref{Eqn:2})(\ref{Eqn:3})(\ref{Eqn:4}), $P$ and $N$ can be eliminated and get following equation
\begin{eqnarray}
	F &=& (M+m)\ddot{x} + ml\ddot{\theta}*\cos{\theta} - ml\dot{\theta}^2\sin{\theta} \, , \nonumber \\ 
	mgl\sin{\theta} &=& (I+ml^2)\ddot{\theta} + ml\ddot{x}\cos{\theta} + c\dot{\theta}.
\end{eqnarray}

\subsection{Linear quadratic regulator}
\begin{figure}[!htb]
	\centerline{
		\psfig{figure=picture/lqr.png,width=3.0in}
	}
	\caption{dynamic system with feedback gain K}
	\label{Figure:LQR}
\end{figure}
The Linear Quadratic Regular (LQR) problem is a canonical problem in the theory of optimal control, 
partially due to the fact that it has analytical solutions that can be derived using a variety of methods, 
and from the fact that LQR is an extremely useful tool in practice.\\
The continuous-time LQR problem is formulated as\cite{kamien2012dynamic}
\begin{eqnarray}	
	&\min_{u(t)}& {\frac{1}{2}x^{T}(t_f)Q(t_f)x(t_f)+\frac{1}{2}\int_{t_f}^{0}[x^{T}(t)Q(t)x(t)+u^{T}(t)R(t)u(t)]} \, , \nonumber \\ 
	&s.t.& \dot{x}(t) = A(t)x(t)+B(t)u(t). \label{Eqn:6}
\end{eqnarray}
where $R$ is a real, symmetric, positive-definite matrix and $Q$ is a real, symmetric, positive semi-definite matrix.
And we define the cost function and the minimum cost function
\begin{equation}
	J(x(t),u(t),t) = l(x(t_f),t)+\int_{0}^{t_f}l(x(t),u(t),t)\mathrm{d}t.	\label{Eqn:7}
\end{equation}
\begin{equation}
	J^*(x(t),t) = \min_{u(t)}{J(x(t),u(t),t)}.	\label{Eqn:8}
\end{equation}
According to the optimal principle in dynamic system, the terminal of the optimal trajectory is still optimal. We can get the following equation
\begin{equation}
	J^*(x(t),t) = \min_{u(t)}{\{\int_{t}^{t'} l(x(\gamma),u(\gamma),\gamma)\mathrm{d}\gamma+J^*(x(t'),t')\}}.	\label{Eqn:9}
\end{equation}
Assume that the cost function is continuous and differentiable, and we can get its Taylor expand
\begin{equation}
	J^*(x',t') = J^*(x,t)+\frac{\partial J^{*T}(x,t)}{\partial x}\dot{x}\Delta t + \frac{\partial J^*(x,t)}{\partial t}\Delta t + o(\Delta t^2) 	\label{Eqn:10}
\end{equation}
Next, we can assume $\Delta t \to 0$, and take equation(\ref{Eqn:10}) into equation(\ref{Eqn:9})
\begin{equation}
	J^*(x,t) = \min_{u(t)}{\{l(x(t),u(t),t)\mathrm{d}t+J^*(x,t)+\frac{\partial J^{*T}(x,t)}{\partial x}\dot{x}\Delta t + \frac{\partial J^*(x,t)}{\partial t}\Delta t\}}.	\label{Eqn:11}
\end{equation}
simplify the equation(\ref{Eqn:11}), and get the famous Hamilton-Jacobi-Bellman equation
\begin{equation}
	0 = J_t^*(x,t) + \min_{u(t)}{\{l(x(t),u(t),t)+J_{x}^{*T}(x,t)\dot{x}\}}.	\label{Eqn:12}
\end{equation}
We now use these to solve the continuous LQR problem. From the equation\ref{Eqn:12}, we calculate $u^*(t)$ of the minimum cost function
\begin{equation}
	\frac{\partial (\frac{1}{2}x^TQx +\frac{1}{2}u^TRu+J_{x}^{*T}(x,t)(Ax+Bu))}{\partial u}=0 \Longrightarrow u^{*} = -R^{-1}B^TJ^*_x .	\label{Eqn:13}
\end{equation}
Since the HJB equation in this case is a first-order partial differential equation of the minimum cost, we need to guess a solution, which we assume to be quadratic:
\begin{equation}
	J^*(x(t),t) = \frac{1}{2}x^T(t)K(t)x(t).	\label{Eqn:14}
\end{equation}
where $K(t)$ is a symmetric positive-definite matrix.\\
Plugging these into the HJB equation we get the Riccati equation:
\begin{equation}
	0 = \frac{1}{2}x^T[\dot K + Q -K^TBR^{-1}B^TK+2KA]x \Longrightarrow 0 = \dot K + Q -K^TBR^{-1}B^TK+2KA. \label{Eqn:15} 
\end{equation}
When solved for $K(t)$ using a specialized Riccati solver, the optimal control law is given by
\begin{equation}
	u^{*}(t) = -R^{-1}B^T(t)K(t)x(t) \label{Eqn:16} 
\end{equation}

\section{Numerical Methods}
\subsection{simulation of inverted pendulum system}
\subsubsection{ODE}
In the last chapter, the inverted pendulum system has been formulated as follows
\begin{eqnarray}
	F &=& (M+m)\ddot{x} + ml\ddot{\theta}\cos{\theta} - ml\dot{\theta}^2\sin{\theta} \, , \nonumber \\ 
	mgl\sin{\theta} &=& (I+ml^2)\ddot{\theta} + ml\ddot{x}\cos{\theta} + c\dot{\theta}. \label{Eqn:17}
\end{eqnarray} 
\\
However, from equation(\ref{Eqn:17}), we can't explicitly know how displacement $x$ and angle $\theta$ change over time.
If we want to simulate this dynamic system, untying the couple of $\ddot{x}$ and $\ddot{\theta}$ is a must.
For that reason, we turn the equation(\ref{Eqn:17}) to
\begin{eqnarray}
	\ddot{x} &=& \frac{mlc\dot{\theta}\cos{\theta} + (I+ml^2)ml\dot{\theta}^2\sin{\theta} -m^2l^2g\cos{\theta}\sin{\theta} + (I+ml^2)F}   {(I+ml^2)(M+m) - m^2l^2\cos^2{\theta}} \, , \nonumber \\ 
	\ddot{\theta} &=& \frac{(M+m)mgl\sin{\theta} -m^2l^2\dot{\theta}^2\sin{\theta}\cos{\theta} -(M+m)c\dot{\theta}-mlF}   {(I+ml^2)(M+m) - m^2l^2\cos^2{\theta}}. \label{Eqn:18}
\end{eqnarray} 
Now, we face a higher order initial value problem. Futher, we can consider this system as a system of 4 ODE's with 4 initial conditions.
Define $ [x_1,x_2,x_3,x_4] = [x,\dot x,\theta, \dot \theta]$
\begin{eqnarray}
	\dot{x_1} &=& x_2 \, , \nonumber \\
	\dot{x_2} &=& \frac{mlcx_4\cos{x_3} + (I+ml^2)mlx_4^2\sin{x_3} -m^2l^2g\cos{x_3}\sin{x_3} + (I+ml^2)F}   {(I+ml^2)(M+m) - m^2l^2\cos^2{x_3}} \, , \nonumber \\ 
	\dot{x_3} &=& x_4 \, , \nonumber \\
	\dot{x_4} &=& \frac{(M+m)mgl\sin{x_3} -m^2l^2x_4^2\sin{x_3}\cos{x_3} -(M+m)cx_4-mlF}   {(I+ml^2)(M+m) - m^2l^2\cos^2{x_3}}. \label{Eqn:19}
\end{eqnarray} 
\subsubsection{2nd order Runge-Kutta Methods}
We use 2nd order Runge-Kutta method\cite{burden2011numerical} to calculate $w = [x,\dot x,\theta, \dot \theta]$, and the scheme is as follows
\begin{eqnarray}
	w_{i+\frac{1}{2}} &=& w_{i}+\frac{h}{2}f(t_i,w_i) \, , \nonumber \\
	w_{i+1} &=& w_i+hf(t_{i+\frac{1}{2}},w_{i+\frac{1}{2}}). \label{Eqn:23}
\end{eqnarray}
In order to ensure the convergence, we analyze consistence and 0-stability seperately.
We analyze the consistence first
\begin{eqnarray}
\frac{\tau_{i+1}}{h} = \frac{y_{i+1}-w_{i+1}}{h} &=& {y_i}'+ \frac{h}{2}{y_i}'' + O(h^2)-f(t+\frac{h}{2},w_i+\frac{h}{2}f(t_i,w_i))  \nonumber \\
& =& {y_i}'+\frac{h}{2}(f_t+f_wy')+O(h^2)-f-\frac{h}{2}f_t-\frac{h}{2}f_wf+O(h^2)   \nonumber\\
& =& O(h^2)  \nonumber
\end{eqnarray}
Then analyze the 0-stability. In real world, $\dot \theta$ exist the maximum. It's easy to prove when $|w_{i,4}| \le C $, $|f(t,w)-f(t,\bar{w})| \le L|w-\bar{w}|$. Since
\begin{eqnarray}
	&&|f(t+\frac{h}{2},w+\frac{h}{2}f(t,w))-f(t+\frac{h}{2},\bar w+\frac{h}{2}f(t,\bar w))|   \nonumber\\
	&\le & L|w+\frac{h}{2}f(t,w)-\bar w-\frac{h}{2}f(t,\bar w))|  \nonumber \\
	&\le& hL|w-\bar w|+\frac{Lh}{2}|f(t,w)-f(t,\bar w)|  \nonumber \\
	&\le& (L+\frac{hL^2}{2})|w-\bar w| \nonumber
\end{eqnarray}
2nd order Runge-Kutta method is 0-stable and consistent, so it will convergent unless $\dot \theta \to \infty$.

\subsection{solution of LQR algorithm}
\subsubsection{ODE}
In the last chapter, we turn the LQR solution of optimal control law into solution of Riccati equation. However, this algorithm can only apply
to linear systems, like formulation (\ref{Eqn:6}). Naturally, we approximate this system as a linear system, which is linearized near the equilibrium point.
\\
Near the balance point, $\theta \approx 0, \sin{\theta} \approx 0, \cos{\theta} \approx 1$. We can linearize the system formualtion(\ref{Eqn:17})
\begin{eqnarray}
	F &=& (M+m)\ddot{x} + ml\ddot{\theta} \, , \nonumber \\ 
	mgl\theta &=& (I+ml^2)\ddot{\theta} + ml\ddot{x} + c\dot{\theta}. \label{Eqn:20}
\end{eqnarray} 
Then we untie the couple of $\ddot{x}$ and $\ddot{\theta}$ from equation(\ref{Eqn:20})
\begin{eqnarray}
\ddot{x} &=& \frac{mlc\dot{\theta} -m^2l^2g\theta + (I+ml^2)F}   {(I+ml^2)(M+m) - m^2l^2} \, , \nonumber \\ 
\ddot{\theta} &=& \frac{(M+m)mgl\theta -(M+m)c\dot{\theta}-mlF}   {(I+ml^2)(M+m) - m^2l^2}. \label{Eqn:21}
\end{eqnarray}
Next, we reduce the order of ODE equation(\ref{Eqn:21}) by increasing the number of variables. Define $ [x_1,x_2,x_3,x_4] = [x,\dot x,\theta, \dot \theta]$
\begin{eqnarray}
	\dot{\begin{bmatrix}
		x_1 \\
		x_2 \\
		x_3 \\
		x_4
		\end{bmatrix}} 
		=
		\frac{\begin{bmatrix}
		 0 & 1 & 0 & 0\\
		 0 & 0 & -m^2l^2g & mlc\\
		 0 & 0 & 0 & 1\\
		 0 & mlb & (M+m)mgl & -(M+m)c
		\end{bmatrix}}{(M+m)(I+ml^2)-m^2l^2}
		\begin{bmatrix}
		x_1 \\
		x_2 \\
		x_3 \\
		x_4
		\end{bmatrix}
		+\frac{\begin{bmatrix}
		0 \\
		I+ml^2 \\
		0 \\
		-ml
		\end{bmatrix}}{(M+m)(I+ml^2)-m^2l^2}F \label{Eqn:22} 
\end{eqnarray}
Now, we obtain the matrix $A$ and matrix $B$ in Riccati equation(\ref{Eqn:15}). If we solve this Riccati equation, we can directly get the optimal control law.
\begin{eqnarray}
	\dot K  = - Q + K^TBR^{-1}B^TK - 2KA 
\end{eqnarray}
where
\begin{eqnarray}
	A = \frac{\begin{bmatrix}
		0 & 1 & 0 & 0\\
		0 & 0 & -m^2l^2g & mlc\\
		0 & 0 & 0 & 1\\
		0 & mlb & (M+m)mgl & -(M+m)c
	   \end{bmatrix}}{(M+m)(I+ml^2)-m^2l^2} &,&
	B = \frac{\begin{bmatrix}
		0 \\
		I+ml^2 \\
		0 \\
		-ml
		\end{bmatrix}}{(M+m)(I+ml^2)-m^2l^2}
\end{eqnarray}
\subsubsection{2nd order Runge-Kutta Methods}
Because for dynamic programming, what we know is the initial value of end point, so we need to reverse the 
integral from $K(T) = 0$. The numerical scheme is as follows
\begin{eqnarray}
	w_{i-\frac{1}{2}} &=& w_{i}-\frac{h}{2}f(t_i,w_i) \, , \nonumber \\
	w_{i-1} &=& w_i-hf(t_{i-\frac{1}{2}},w_{i-\frac{1}{2}}). \label{Eqn:24}
\end{eqnarray}
The proof of convergence in 2nd order Runge-Kutta method is mentioned in 3.1.2. Now, we discuss the situation that $f(t,w)$ satisfy Lipschitz continuity.
\begin{eqnarray}
	\frac{\partial (- Q + K^TBR^{-1}B^TK - 2KA)}{\partial K} =2K^TBR^{-1}B^T-2A \nonumber
\end{eqnarray}
If $|K| \le \infty$ all the time, this numerical scheme will convergent.

\section{Results}
This project implements in python, and dynamic results with more detail can be seen example.gif.
\subsection{simulation of dynamic system with zero-input}
In this part, we simulate the inverted pendulum system without force F, because we can visually observe the 
effect of our numerical scheme through the law of energy conservation. 
\subsubsection{example1}
In the first example, I set the rotational friction coefficient $c=0$ and the force of cart $F=0$. In 
this condition, the inverted pendulum system observe the energy conservation. In other word, this system are supposed 
to show periodicity and symmetry.
\begin{figure}[!hbt]
	\centering
	\mbox{
	(a)\subfigure{\psfig{figure=picture/init_3.png,width=1.8in}}
	(b)\subfigure{\psfig{figure=picture/trajectory_1.png,width=2.0in}}
	(c)\subfigure{\psfig{figure=picture/angle_1.png,width=2.0in}}
	}
	\vspace{15pt}
	\caption{$T=10, c=0, F(t)=0, x(0) = 0, \theta(0)=0.1$}
	\label{Figure:example1}
\end{figure}
\subsubsection{example2}
In the second example, we consider the rotational friction coefficient $c\ne 0$.
In this condition, the energy of inverted pendulum system gradually decays, and system are supposed to reach the equilibrium point $\theta = \pi$.
\begin{figure}[!hbt]
	\centering
	\mbox{
	(a)\subfigure{\psfig{figure=picture/init_3.png,width=1.8in}}
	(b)\subfigure{\psfig{figure=picture/trajectory_2.png,width=2.0in}}
	(c)\subfigure{\psfig{figure=picture/angle_2.png,width=2.0in}}
	}
	\vspace{15pt}
	\caption{$T=10, c=0.1, F(t)=0, x(0) = 0, \theta(0)=0.1$}
	\label{Figure:example2}
\end{figure}
\subsection{LQR algorithm to keep balance}
In this part, we get the force of cart $F(t)=-R^{-1}B^T(t)K(t)x(t)$ through the optimal control law $K$ and current system state varables $x$.
We implement this force to the original system to stabilize the system at an unstable equilibrium point $\theta = 0$.
\subsubsection{example3}
In the third eaxmple, we care about the angle of the stick and the velocity of the cart, which are supposed to stabilize in $\theta = 0,\dot{x} = 0$.
In this condition, we set 
$Q = \begin{bmatrix}
	0 & 0 &0  & 0\\
	0 & 1 &0  & 0\\
	0 & 0 & 1 & 0\\
	0 & 0 & 0 & 1
   \end{bmatrix}, R = 1$.
\begin{figure}[!hbt]
	\centering
	\mbox{
	(a)\subfigure{\psfig{figure=picture/init_3.png,width=1.8in}} 
	(b)\subfigure{\psfig{figure=picture/trajectory_3.png,width=2.0in}}
	(c)\subfigure{\psfig{figure=picture/angle_3.png,width=2.0in}}
	}
	\vspace{15pt}
	\caption{$T=10, c=0.0, x(0) = 0, \dot{x}(0) = 0,\theta(0)=0.5, \dot{\theta}(0)=0$}
	\label{Figure:example3}
\end{figure}

\subsubsection{example4}
In the forth eaxmple, we not only care about the angle and the velocity, but also the location where the cart finally stay. 
This means pendulum system are supposed to stabilize in $\theta = 0, x = 0$. In this condition, we set 
$Q = \begin{bmatrix}
	1 & 0 &0  & 0\\
	0 & 1 &0  & 0\\
	0 & 0 & 1 & 0\\
	0 & 0 & 0 & 1
   \end{bmatrix}, R = 1$
, and give the cart an initial speed.
\begin{figure}[!hbt]
	\centering
	\mbox{
	(a)\subfigure{\psfig{figure=picture/init_4.png,width=1.8in}}
	(b)\subfigure{\psfig{figure=picture/trajectory_4.png,width=2.0in}}
	(c)\subfigure{\psfig{figure=picture/angle_4.png,width=2.0in}}
	}
	\vspace{15pt}
	\caption{$T=10, c=0.0, x(0) = -2, \dot{x}(0) = 2,\theta(0)=1, \dot{\theta}(0)=0$}
	\label{Figure:example4}
\end{figure}

\section{Conclusions}
In this project, I first model an inverted pendulum system, then transforme these equation into the familiar ODE, and finally solved it with the 2nd order Runge-Kutta method.
What interests me most is how to solve equations stably and accurately simulate a dynamic system.
\\
However, there are also some remaining issues in this project. For example, since we approximate the system as a linear system in control law design, 
the LQR algorithm fails when the pendulum moves away from the equilibrium point, like picture(\ref{Figure:example5})~
\begin{figure}[!hbt]
	\centering
	\mbox{
	(a)\subfigure{\psfig{figure=picture/init_5.png,width=1.8in}}
	(b)\subfigure{\psfig{figure=picture/trajectory_5.png,width=2.0in}}
	(c)\subfigure{\psfig{figure=picture/angle_5.png,width=2.0in}}
	}
	\vspace{15pt}
	\caption{$T=10, c=0.0, x(0) = 0, \dot{x}(0) = 0,\theta(0)=1.5, \dot{\theta}(0)=0$}
	\label{Figure:example5}
\end{figure}


% \begin{thebibliography}{10}
% \bibitem{baker90}{\sc G.~L. Baker, \& J.~P. Gollub}, {\em Chaotic Dynamics}, Cambridge University Press, 1990.
% \bibitem{lamport94}{\sc L.~Lamport}, {\em \LaTeX: A document preparation system}, Addison Wesley Publishing Company, 1994.
% \end{thebibliography}
\bibliographystyle{plain}
\bibliography{ref}
\end{document}
